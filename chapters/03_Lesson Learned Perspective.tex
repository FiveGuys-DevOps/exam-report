\subsection*{Development and Refactoring}

During development, we encountered a problem with Vagrant. Specifically, Vagrant was not working on other systems other than Ubuntu. Therefore, due to this limitation, we have decided to manually create the initial setup, which included creating the droplet and installing initial dependencies, such as docker and docker-compose. Even though, the solution worked in this project due to the scale, it would be optimal to use Vagrant to automatically create the initial setup for the project. This is especially true for larger scale projects.

\subsection*{Operation and Maintenance}

We encountered two problems when implementing logging during the maintenance phase.
Firstly, we did not rotate the log files at the beginning, so the single log file became big enough that our Droplet did not have enough space and stopped working.\\
This can be seen in the simulator plots as we have a drop on the 28th of March. Additionally, this has also resulted in us getting more errors because the users that were supposed to be created when the server was down, were not created, so when the simulator attempts to have them follow or twit more errors occur. The solution was to scale up the size of the droplet and introduce rotating log files with max-size to prevent this from happening.\\
Secondly, we did not check the log formatting. The initial implementation was done based on gathering every data point we needed, therefore the logs had the following formatting:
\begin{center}
    \{asctime\} \{levelname\} \{module\} \{process:d\} \#\{thread:d\} \{message\}\\
    2023-05-09 12:47:56,545 WARNING log 11 139829591119424 Not Found: /sim/fllws/Florentino
\end{center}
However, when implementing the Django-log-viewer, the requested formatting to display the logs in the proper way is:

\begin{center}
    [\{levelname\}] \{asctime\} \{name\}: \{message\}\\
    \{[INFO]\} 2023-05-16 11:04:51,008 django.server: "GET /public HTTP/1.1" 200 1037
\end{center}

Therefore, since we did not use the correct formatting, when accessing the Django-log-viewer, the logs are in one extensive line, instead of being on separate lines.
\\\\
Another encountered difficulty is that the number of equal tweets on GUI and API is zero. We concluded that the reason for this was because our GUI took too long to load, and in the time it was loading, the simulator already pushed multiple tweets, therefore never achieving the parity. This is an unresolved problem in this project's MiniTwit.