Throughout this course we were given the task of continuously improving and adding features to a legacy code base. The code base in question was an aging mini version of Twitter called \textit{Minitwit}. 

\section*{Technologies}

\subsection*{Refactoring}

Minitwit was originally written in Python 2 using Flask. The goal was to get the website running on a modern framework, this was done by firstly using the library 2to3, which converts python 2 code into python 3 code. Lastly to complete the refactoring we rewrote the code to run using the Django framework.

\subsection*{Containerized}

Containerized is a term in software development, which references the fact that you isolate part of your software into containers. To implement this technology we decided to use Docker. Using such technologies is beneficial when writing in teams, since you can keep all the moving parts isolated, thus solving the exponentially difficult problem of dependencies. By forcing everyone to use the dependencies within the container, this enables the container to be run on any computer/server having docker installed.
\\\\
We setup a total of 4 Docker containers each responsible for one specific task as seen below:


\begin{enumerate}
    \item Django server - one of the things we ar ehosting the ip
    \item Database - postgress container 
    \item Prometheus container, collecting data from django container, how many requests monitoring
    \item Graphana visualization of permformance e.i. monitoring
\end{enumerate}

\subsection*{Continuous Integration and Continuous Deployment}
Devops is a way development method where the time from the testing environment to production is rapidly shortened using a lot of automation to ensure high quality. Since we are using GitHub as our version control tool, we can use build in system for testing before deploying. GitHub has a build in CI/CD tool where you can build a pipeline of tools which has to be completed for the branches to be merged.



\subsection*{Cloud}

Since we are creating a web app, it has to be hosted somewhere, since we do not want our development computers to function as the server. By having a centralized structure, that is a singular place for hosting, we only had to know one provider, and everyone could connect to it from anywhere. We used a Infrastructure-as-a-Service (IaaS) provider called Digital Ocean, due to them having a nice UX-design with easy to scale both horizontally and vertically.


\subsection*{Monitoring}
Monitoring is simply the act of getting information about the system. The hard part is what to monitor since it doesn't make sense to log and measure every part of the system. Therefor if you select the correct metrics, it can give you an unprecedented amount of insight. It can be anything from the technical aspect of uptime/availability, to more subjective things such as how the users rate your service. The tool of choice for this task were Grafana for *INSERT* tasks and Silk for *INSERT*.


\subsection*{Software quality}

When writing a piece of software, one can do it fast or one can do it right. If you write something fast to get it to production, then you take on a bit of \textit{debt}, which in it self is not bad as long as it is paid back. This term of writing something fast at the expense of quality is called technical debt. \\
The flawed problem lies in the definition which build upon the term software quality. This quality can be hard to measure, however one can use guidelines and tools when writing to assist the developer in making the right decisions. These tools which are analyzing your code before executing it is referred to as static analysis tools. \\
We used 3 static analysis tools:
\begin{enumerate}
    \item SonarCloud performes a bunch of actions. It is implemented directly into our CI/CD pipeline, where it shows bugs amongst other things. 
    \item Snyk is a tool which shows potential vulnerabilities in the code. This can be anything from creating pull requests on GitHub to update dependencies, to showing where in the code these vulnerabilities lies.
    \item Megalinter provides consistency to our code. Since there are bunch of ways of writing the same code that are all syntactically correct, this tool creates consistency throughout our code base.
\end{enumerate}

\subsection*{Logging}
Logging has been implemented using Django's logging. We decided to use what Django offered as firstly, implementation and integration-wise these are robust enough and offer the same functionality as other third-party tools. In addition, we tried to limit the extensive use of third-party tools as these can lead to potential security risks.
\subsection*{Pentest}
When it comes to handling something valuable, there will always be people trying to get their hands on it.\\
In our case, it can come in the form of infiltrating our system in order to acquire confidential information regarding our users or even hijacking a user's account directly.\par
In order to discover what possible vulnerabilities are possibly exploitable in our system, a series of Pen Tests (Short for \textit{Penetration Testing}) were executed.\\
The tool used to conduct this testing was \textbf{ZAP} (\textit{Zed Attack Proxy}), in 

\subsection*{Swarm}

\subsection*{Terraform}


\section*{Current state}

Here we should descipe the current state, that is if we have bugs and so on. We should use some static analysis tools for this section


\section*{Licence}

The licence of a software is important due to it dictating how external users can use your software. A limitation on this section is we only look at the python packages using the following library \textbf{pip-licenses} and command \textbf{ pip-licenses --order=license}. This returns a list of all the python packages installed and the corresponding licence as seen in table \ref{tab:dependencies}

Since we are using libraries which are under the GNU licence, our software becomes derivative work of the dependencies and thus warrants us to also use the GNU licence and publish the changes. 



\begin{table}[ht]
\centering
\scalebox{0.8}{%
\begin{tabular}{|c|c|c|}
\hline
\textbf{Name}       & \textbf{Version} & \textbf{License}                                        \\ \hline
distro              & 1.7.0            & Apache Software License                                 \\ \hline
django-prometheus   & 2.2.0            & Apache Software License                                 \\ \hline
importlib-metadata  & 4.6.4            & Apache Software License                                 \\ \hline
prometheus-client   & 0.7.1            & Apache Software License                                 \\ \hline
tzdata              & 2022.7           & Apache Software License                                 \\ \hline
cryptography        & 3.4.8            & Apache Software License; BSD License                    \\ \hline
Django              & 4.1.7            & BSD License                                             \\ \hline
SecretStorage       & 3.3.1            & BSD License                                             \\ \hline
asgiref             & 3.6.0            & BSD License                                             \\ \hline
oauthlib            & 3.2.0            & BSD License                                             \\ \hline
sqlparse            & 0.4.4            & BSD License                                             \\ \hline
python-apt          & 2.4.0+ubuntu1    & GNU GPL                                                 \\ \hline
PyGObject           & 3.42.1           & GNU Lesser General Public License v2 or later (LGPLv2+) \\ \hline
gprof2dot           & 2022.7.29        & GNU Lesser General Public License v3 or later (LGPLv3+) \\ \hline
launchpadlib        & 1.10.16          & GNU Library or Lesser General Public License (LGPL)     \\ \hline
lazr.restfulclient  & 0.14.4           & GNU Library or Lesser General Public License (LGPL)     \\ \hline
lazr.uri            & 1.0.6            & GNU Library or Lesser General Public License (LGPL)     \\ \hline
psycopg2            & 2.9.6            & GNU Library or Lesser General Public License (LGPL)     \\ \hline
wadllib             & 1.3.6            & GNU Library or Lesser General Public License (LGPL)     \\ \hline
PyJWT               & 2.3.0            & MIT License                                             \\ \hline
autopep8            & 2.0.2            & MIT License                                             \\ \hline
blinker             & 1.4              & MIT License                                             \\ \hline
dbus-python         & 1.2.18           & MIT License                                             \\ \hline
django-log-viewer   & 1.1.7            & MIT License                                             \\ \hline
django-silk         & 5.0.3            & MIT License                                             \\ \hline
httplib2            & 0.20.2           & MIT License                                             \\ \hline
jeepney             & 0.7.1            & MIT License                                             \\ \hline
more-itertools      & 8.10.0           & MIT License                                             \\ \hline
pycodestyle         & 2.10.0           & MIT License                                             \\ \hline
pyparsing           & 2.4.7            & MIT License                                             \\ \hline
six                 & 1.16.0           & MIT License                                             \\ \hline
tomli               & 2.0.1            & MIT License                                             \\ \hline
zipp                & 1.0.0            & MIT License                                             \\ \hline
keyring             & 23.5.0           & MIT License; Python Software Foundation License         \\ \hline
distro-info         & 1.1build1        & UNKNOWN                                                 \\ \hline
unattended-upgrades & 0.1              & UNKNOWN                                                 \\ \hline
\end{tabular}%
}
\caption{All Dependencies}
\label{tab:dependencies}
\end{table}
                           


\section*{Architecture}

$$\frac{balls}{zazaaa}$$
$$\frac{Kiumsht}{flmi}$$